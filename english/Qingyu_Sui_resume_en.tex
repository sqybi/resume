%% start of file
%% Copyright 2006-2013 Xavier Danaux (xdanaux@gmail.com).
%
% This work may be distributed and/or modified under the
% conditions of the LaTeX Project Public License version 1.3c,
% available at http://www.latex-project.org/lppl/.


\documentclass[10pt,a4paper,roman]{moderncv} % possible options include font size ('10pt', '11pt' and '12pt'), paper size ('a4paper', 'letterpaper', 'a5paper', 'legalpaper', 'executivepaper' and 'landscape') and font family ('sans' and 'roman')

% moderncv themes
\moderncvstyle{classic} % style options are 'casual' (default), 'classic', 'oldstyle' and 'banking'
\moderncvcolor{blue} % color options 'blue' (default), 'orange', 'green', 'red', 'purple', 'grey' and 'black'
%\renewcommand{\familydefault}{\sfdefault} % to set the default font; use '\sfdefault' for the default sans serif font, '\rmdefault' for the default roman one, or any tex font name
\nopagenumbers{} % uncomment to suppress automatic page numbering for CVs longer than one page

% character encoding
%\usepackage[utf8]{inputenc} % if you are not using xelatex ou lualatex, replace by the encoding you are using
%\usepackage{CJKutf8} % if you need to use CJK to typeset your resume in Chinese, Japanese or Korean

% set fonts
%\setmainfont{Palatino Linotype}
%\setsansfont{Helvetica}

% adjust the page margins
\usepackage[perpage,bottom,stable]{footmisc}
\usepackage[vmargin=1cm,hmargin=1.5cm]{geometry}
\setlength{\hintscolumnwidth}{2.5cm} % if you want to change the width of the column with the dates
%\setlength{\makecvtitlenamewidth}{10cm} % for the 'classic' style, if you want to force the width allocated to your name and avoid line breaks. be careful though, the length is normally calculated to avoid any overlap with your personal info; use this at your own typographical risks...

% personal data
\name{Qingyu}{Sui}
%\address{\#401, Zhichundongli, Zhichun Road, Haidian District}{Beijing, China}{China} % optional, remove / comment the line if not wanted; the "postcode city" and "country" arguments can be omitted or provided empty
\phone[mobile]{+86~186~0057~4747} % optional, remove / comment the line if not wanted; the optional "type" of the phone can be "mobile" (default), "fixed" or "fax"
\email{sqybilly@gmail.com} % optional, remove / comment the line if not wanted
\social[linkedin]{sqybi} % optional, remove / comment the line if not wanted
\social[github]{sqybi} % optional, remove / comment the line if not wanted

%\title{Resumé title} % optional, remove / comment the line if not wanted
%\phone[fixed]{+2~(345)~678~901}
%\phone[fax]{+3~(456)~789~012}
%\homepage{sqybi.com} % optional, remove / comment the line if not wanted
%\social[twitter]{sqybi} % optional, remove / comment the line if not wanted
%\extrainfo{additional information} % optional, remove / comment the line if not wanted
%\photo[64pt][0.4pt]{picture} % optional, remove / comment the line if not wanted; '64pt' is the height the picture must be resized to, 0.4pt is the thickness of the frame around it (put it to 0pt for no frame) and 'picture' is the name of the picture file
%\quote{Some quote} % optional, remove / comment the line if not wanted

% to show numerical labels in the bibliography (default is to show no labels); only useful if you make citations in your resume
%\makeatletter
%\renewcommand*{\bibliographyitemlabel}{\@biblabel{\arabic{enumiv}}}
%\makeatother
%\renewcommand*{\bibliographyitemlabel}{[\arabic{enumiv}]}% CONSIDER REPLACING THE ABOVE BY THIS

% bibliography with mutiple entries
%\usepackage{multibib}
%\newcites{book,misc}{{Books},{Others}}
%----------------------------------------------------------------------------------
%            content
%----------------------------------------------------------------------------------
\begin{document}
%\begin{CJK*}{UTF8}{gbsn}                          % to typeset your resume in Chinese using CJK
%-----       resume       ---------------------------------------------------------
\makecvtitle

\vspace{-1cm}

\section{Education}
\cventry{2009 -- 2013}{B.S. of Computer Science and Engineering}{Shanghai Jiao Tong University (SJTU)}{}{}{}

\section{Experience}
\cventry{2013 -- present}
        {Software Development Engineer}
        {}
        {Microsoft}
        {}
        {\begin{itemize}%
         \item Spartan Team, ASG.
               \begin{itemize}%
               \item Execution Broker Engine for Metadata Platform\newline{}%
                     An engine which takes some requirements of moving data from one storage to another, then execute and monitor them on an internal job scheduling platform. This is for Office 365 Big Data Platform and some other internal customers.\newline{}%
                     Responsible for the design and implementation of whole engine.
               \end{itemize}
         \item (Cortana) Proactive Experience Team, Information Platform and Experiences, STC.
               \begin{itemize}%
               \item Cortana Proactive Canvas Experience\newline{}%
                     Responsible for some unpublished secret features on Cortana proactive canvas.
               \item Diagnostic Dashboard for Cortana\newline{}%
                     A diagnostic infrastructure for Cortana developers with user-friendly frontend.\newline{}%
                     Responsible for some of the end-to-end debug scenarios, including log collection, log processing, user interface (HTML and Javascript/JQuery) and so on.
               \end{itemize}
         \item Activity Storage Team, MSN Social and Community Platform, STC.
               \begin{itemize}%
               \item Image Storage\newline{}%
                     The image storage backend for Bing Score.\newline{}%
                     Responsible for a performance test framework on Azure for this storage.
               \item Activity Storage\newline{}%
                     A no-SQL storage as the backend of MSN, Bing Score, etc., aimed for a real-time in-memory storage service.\newline{}%
                     Responsible for evaluation and improvement of performance.
               \end{itemize}
         \end{itemize}}
\cventry{2013}
        {Research Intern}
        {Visulization Group, Internet Graphics}
        {Microsoft Research Asia}
        {}
        {Visualization for large-scale evolving text streams based on rose tree.\footnotemark[1]{}}
        \footnotetext[1]{How Hierarchical Topics Evolve in Large Text Corpora, \textit{IEEE InfoVis 2014}}
\cventry{2012}
        {SDE Intern}
        {SQL Dev Group, STB}
        {Microsoft}
        {}
        {Responsible for implementing the NoInit feature in Sync Framework (Data Sync Service), to speed up the initial sync process.}
\cventry{2012}
        {Popular Science Writer}
        {Guokr\footnotemark[2]{}}
        {}
        {}
        {Write popular science articles\footnotemark[3]{} of Math and Informatics area for Guokr.com (in Chinese).}
        \footnotetext[2]{\link{http://www.guokr.com/}}
        \footnotetext[3]{\link{http://www.guokr.com/i/0492216944/articles/}}
\cventry{2010}
        {Teaching Assistant}
        {Data Structure}
        {SJTU}
        {}
        {Worked as the teaching assistant of the data structure course in SJTU.\newline{}%
         Responsible for grading, discussion and taking over class for absent instructor\footnotemark[4]{}, etc.}
        \footnotetext[4]{Instructor: Prof. Yu (\link{http://apex.sjtu.edu.cn/apex\_wiki/yyu})}
%\cventry{2009}{队员}{ACM/ICPC}{上海交通大学}{}{在2009年参加上海交通大学ACM队。}
%\cventry{2008 -- 2009}{Teacher}{Xindi Education}{Tianjin}{}{Worked as teacher of Olympiad in Informatics (OI) in Xindi Education.}
%\cventry{2011}
%        {Developer}
%        {Undergraduate Transcript and Certificate of Standing Transaction System}
%        {SJTU Office of Academic Affairs}
%        {}
%        {Developed the online ordering part and the interface between this system and Campus Payment system.\newline{}Used C\# and ASP.NET as the programming language in development.}
%\cventry{2009}
%        {Developer}
%        {PuTao\footnotemark{}}
%        {SJTU Network Center}
%        {}
%        {Responsible for the forum development, using PHP as the programming language. }\footnotetext{\link{http://pt.sjtu.edu.cn/}}
%\cventry{2009}{Tutor}{Shanghai Experimental High School}{Shanghai}{}{Worked as teacher of Olympiad in Informatics (OI) in Shanghai Experimental High School.}

\section{Publications}
\cventry{2014}
        {Clustering Image Search Results by Entity Disambiguation}
        {ECML/PKDD 2014}{}{}
        {Kaiqi Zhao, Zhiyuan Cai, Qingyu Sui, Enxun Wei and Kenny Zhu}
\cventry{2013}
        {CISC: Clustered Image Search by Conceptualization}
        {EDBT 2013}{}{}
        {Kaiqi Zhao, Enxun Wei, Qingyu Sui, Kenny Zhu and Eric Lo}

%\section{Projects}
%\cventry{2011 -- 2012}
%        {Context Extraction for Name Disambiguation on Structured Web Pages}{Participation in Research Program (PRP)}
%        {Advanced Data And Programming Technology Lab\footnotemark{}}
%        {SJTU}
%        {\begin{itemize}
%          \item{An algorithm which can extract context from a webpage with a specific entity name, used as a preprocessor for name disambiguation methods.}
%          \item{Developed majority of the algorithm, and finished the calibration of the test dataset, as a principal member of this project.}
%         \end{itemize}
%        }
%        \footnotetext{\link[http://www.cs.sjtu.edu.cn/$\sim$kzhu/adapt]{http://www.cs.sjtu.edu.cn/~kzhu/adapt}}
%\cventry{2012 -- present}
%        {Image Classification by Context Conceptualization}
%        {}
%        {Advanced Data And Programming Technology Lab}
%        {SJTU}
%        {\begin{itemize}
%          \item{Classify images in the search result of Google Image by textual information from their original webpages. We use incremental HAC as the clustering algorithm, Sibling and Wikification to extract signals from HTML text.}
%          \item{Responsible for the design and implementation of the core algorithm.}
%         \end{itemize}
%        }
%\cventry{2010 -- 2012}
%        {The Research and Development of Special Acoustic Directional Locator}
%        {The 2nd Shanghai Jiao Tong University Students' Innovative Practice Projects}
%        {}
%        {}
%        {\begin{itemize}
%          \item{A system which can react to special sound and locate the sound source.}
%          \item{Developed the Labview code in the project.}
%         \end{itemize}
%        }

\section{Awards}
\cventry{2009}
        {Google Codejam}
        {Top 500 Place}
        {}{}{}
\cventry{2008}
        {National Olympiad in Informatics (NOI)}
        {46th Place}
        {Silver Medal}
        {}{}
%\cventry{2009}{Youdao Nanti}{2nd Round}{}{}{}

\section{Skills}
\cvitem{Algorithm}
       {ACM/ICPC-level algorithmic problem-solving, Topcoder Algorithm Rating 1605\footnotemark[5]{}.}
       \footnotetext[5]{\link{http://community.topcoder.com/tc?module=MemberProfile\&cr=22692587}}
\cvitem{Language}
       {\textbf{C\#}, \textbf{C++}\newline{}%
        Python, C, PHP, \LaTeX, HTML.}
        %Mathematica, SQL, Java, CSS
%\cvitem{Typography}
%       {Familar with typesetting in \LaTeX and graphics drawing in Asymptote.}
\cvitem{General}
       {Excellent ability of learning programming languages and algorithms, readily accepts new things.}
%        Friendly and easily getting along with other people.}
\end{document}

%% end of file.

